\documentclass[11pt]{article}

\usepackage{xcolor}
\usepackage{fullpage}
\usepackage[colorlinks, allcolors=blue]{hyperref}
\usepackage{listings}
\usepackage{parskip}

\def\trademarksymbolr   {$^{\mbox{\textregistered}}$}

\lstset
{
  basicstyle=\ttfamily,
}

\date{}
\title{\textbf{A~\lstinline{pre_main()} function that runs before\\
                static initialization that precedes \lstinline{main()}}}

\author{}

\newcommand{\emailSD}{\href{mailto:sebastian.davalle@tallertechnologies.com}{sebastian.davalle@tallertechnologies.com}}
\newcommand{\emailDG}{\href{mailto:daniel.gutson@tallertechnologies.com}{daniel.gutson@tallertechnologies.com}}
\newcommand{\emailCK}{\href{mailto:e\_float@yahoo.com}{e\_float@yahoo.com}}


\begin{document}

\maketitle\vspace{-2cm}

\begin{flushright}
  \begin{tabular}{ll}
  Document \#:&D9999\\
  Date:       &\date{\today}\\
  Project:    &Programming Language C++\\
              &Evolution Working Group\\
  Reply-to:   &\author{S\'{e}bastian Davalle}\\
              &\textless\emailSD\textgreater\\
              &\author{Daniel Gutson}\\
              &\textless\emailDG\textgreater\\
              &\author{Christopher Kormanyos}\\
              &\textless\emailCK\textgreater\\
  \end{tabular}
\end{flushright}

\section{Motivation}
\subsection*{Overview}
Many systems require user-specific initialization. In particular,
deeply embedded environments may have user-specific initialization
that should be executed as soon as possible following power-on-reset
of the microcontroller.
Typical examples thereof include initialization of
I/O~ports, watchdog timers, instruction and data cache systems,
interrupt systems, clock systems,~etc. For these environments,
postponing user-specific initialization until~\lstinline{main()}
may detract from quality of design.
Another important use is setting up tools such as logging or allocator
mechanisms---notoriously meticulous activities that could
potentially be simplified with a \lstinline{pre_main()} initialization.

\section{Proposed Solution}
This document proposes changes in the standard text reflecting
the specification of a \lstinline{pre_main()} function
that is intended to be called prior to static initialization that
precedes \lstinline{main()}.

\subsection*{Modification to the standard text}
\begin{enumerate}

  \item \textbf{[basic.start.main] Add a new paragraph \S3.6.1p6}

  The \lstinline{pre_main()} function shall be called prior to
  \emph{static} \emph{initialization} (\S3.6.2)
  that precedes \lstinline{main}.
  The linkage (3.5) of \lstinline{pre_main()} is implementation-defined.
  The \lstinline{pre_main()} function is parameter-free.
  The return value of \lstinline{pre_main()} is~\lstinline{void}.
  The \lstinline{pre_main()} function lacks side-effects.
  The mechanism ensuring that \lstinline{pre_main()} lacks
  side-effects is implementation-defined.
  %%%% END OF FIRST ITEM %%%%

\end{enumerate}

\section{Existing workarounds}

We will now investigate existing workarounds for providing
a call mechanism for~\lstinline{pre_main()} or a similar
call mechanism.

\subsection*{Sample startup code}

Embedded systems developers and compiler implementers often
write startup code. In this case, it is straight forward
to support a~\lstinline{pre_main()} function.
For example, we will now look at sample startup
code~\cite{bib:kormanyos2013,bib:kormanyosgit2015}
showing how an implementation could potentially provide
a call mechanism for~\lstinline{pre_main()}.
  
\begin{lstlisting}
void __my_startup()
{
  // Load the sreg register.
  asm volatile("eor r1, r1");
  asm volatile("out 0x3F, r1");

  // Setup the stack pointer.
  asm volatile("ldi r28, lo8(__initial_stack_pointer)");
  asm volatile("ldi r29, hi8(__initial_stack_pointer)");
  asm volatile("out 0x3E, r29");
  asm volatile("out 0x3D, r28");

  // A potential call mechanism for pre_main.
  pre_main();

  // Initialize statics from ROM to RAM.
  // Initialize default-initialized static RAM.
  crt::init_ram();

  // Call all ctor initializations.
  crt::init_ctors();

  // Call main (and never return).
  asm volatile("call main");

  // Catch an unexpected return from main.
  for(;;)
  {
    // Replace with a loud error if desired.
    mcal::wdg::secure::trigger();
  }
}
\end{lstlisting}
  
This example has been taken from the low-level
initialization sequence of a popular 8--bit microcontroller.
The code has been compiled and tested with GCC~4.8.1~\cite{bib:gccwebsite}.
After setting a CPU register, the stack pointer is initialized.
Immediately following stack setup, \lstinline{pre_main()} is called.
Note that \lstinline{pre_main()} is called prior to
static initialization.

\subsection*{Commercially available microcontroller compilers}

Some commercially available microcontroller compilers provide
a custom hook (in the sense of \lstinline{pre_main()}) that
is called before static initialization that precedes \lstinline{main()}.
The IAR~Systems C/C++ compiler and debugger toolchain~\cite{bib:iar2015},
for instance, uses an implementation-specific function
called \lstinline{__low_level_init()} for this purpose.
The user is responsible for supplying the content (if~any)
of~\lstinline{__low_level_init()}.

\section{Future Work}

The motivation and justification for a potential \lstinline{pre_main()}
is analogous in C and C++. Therefore, specifying \lstinline{pre_main()}
could potentially be addressed in~WG14 as well as~WG21.

Along these lines, do we need two versions of \lstinline{pre_main()}?
In particular,

\begin{lstlisting}
void    ::pre_main(void); // Intended for C/C++
void std::pre_main();     // Intended for C++
\end{lstlisting}

What is the proper name of a potential \lstinline{pre_main()}?
Is \lstinline{pre_init()} a better name because it
more clearly reflects when the function is called?

Despite the proposal that \lstinline{pre_main()} lacks side-effects,
it could be beneficial to allow \lstinline{pre_main()} to initialize certain
\emph{clearly} \emph{identifiable} non-local variables
having static storage duration.
Embedded systems tool chains for C/C++ typically provide special
linker sections with implementation-specific names such as
\lstinline{.noinit}, \lstinline{.noclear},~etc. These are meant
to store non-local variables having static storage duration that are not
intended to undergo static initialization.
Attributes such as \lstinline{[[noclear]]} or \lstinline{[[noinit]]}
could be used to clearly identify these.

Consider, for example, the \lstinline{reset_reason} in
the following code.

\begin{lstlisting}
typedef enum enum_reset_reason
{
  power_on_reset,
  watchdog_reset,
  software_reset
}
reset_reason_type;

[[noclear]] reset_reason_type reset_reason;
\end{lstlisting}

\noindent
Here, \lstinline{reset_reason} is intended to be initialized by
\lstinline{pre_main()} in the application, not via conventional
static initialization.

\section{Discussion}

TBD: Summarize the discussion regarding when \lstinline{pre_main()}
should be called and why.

TBD: Summarize the discussion regarding the dangers of
offering an open user-interface that precedes \lstinline{main()}.
Will users run into inordinate amounts of trouble with this
proposed interface?

TBD: Specify what can be done and not inside \lstinline{pre-main()}

TBD: Distinguish the usage in terms of freestanding/hosted implementation.

TBD: Define what should be done in a failure condition. 

TBD: Exception handling (nonexcept attribute) 

TBD: Calls to \lstinline{std::} functions is allowed or not? 

TBD: calls to \lstinline{atexit()},
\lstinline{exit()},
\lstinline{at_quick_exit()},
\lstinline{quick_exit()} should be undefined.

TBD: call to \lstinline{terminate()} should be implementation specific.

\section{Acknowledgments}

TBD: Acknowledge the participants.

\section{References}
\renewcommand{\section}[2]{}%
\begin{thebibliography}{9}


\bibitem{bib:gccwebsite}
Free Software Foundation: \emph{GNU Compiler Collection}\\
\url{http://gcc.gnu.org}~(2015)

\bibitem{bib:iar2015}
IAR~Systems, \emph{IAR Embedded Workbench\trademarksymbolr\
C/C++ compiler and debugger toolchain},\\
\url{http://www.iar.com/Products/IAR-Embedded-Workbench}~(2015)

\bibitem{bib:kormanyos2013}
C.~M.~Kormanyos, \emph{Real-Time C++},
Springer Verlag, Heidelberg,~2013

\bibitem{bib:kormanyosgit2015}
C.~M.~Kormanyos, \emph{real-time-cpp : Companion code for Real-Time C++},
{\href{https://github.com/ckormanyos/real-time-cpp}{real-time-cpp}},~2015

%% TBD %%

\end{thebibliography}

\end{document}
